%%%%%%%%%%%%%%%%%%%%%%%%%%%%%%%%%%%%%%%%%
% Lachaise Assignment
% LaTeX Template
% Version 1.0 (26/6/2018)
%
% This template originates from:
% http://www.LaTeXTemplates.com
%
% Authors:
% Marion Lachaise & François Févotte
% Vel (vel@LaTeXTemplates.com)
%
% License:
% CC BY-NC-SA 3.0 (http://creativecommons.org/licenses/by-nc-sa/3.0/)
% 
%%%%%%%%%%%%%%%%%%%%%%%%%%%%%%%%%%%%%%%%%

%----------------------------------------------------------------------------------------
%	PACKAGES AND OTHER DOCUMENT CONFIGURATIONS
%----------------------------------------------------------------------------------------

\documentclass{article}
\usepackage{hyperref}
\usepackage{amssymb}
\usepackage{graphicx}
\usepackage{pgfplots}
\usetikzlibrary{patterns}
\usepackage{graphicx,subfigure}

\input{structure.tex} % Include the file specifying the document structure and custom commands


\linespread{1.3}
%----------------------------------------------------------------------------------------
%	ASSIGNMENT INFORMATION
%----------------------------------------------------------------------------------------

\title{Assignment01} % Title of the assignment

\author{2017120175 Jae-hong Park\\ \texttt{westsun0920@gmail.com}} % Author name and email address

\date{2018.09.20} % University, school and/or department name(s) and a date



%----------------------------------------------------------------------------------------

\begin{document}



\maketitle % Print the title

\href{https://github.com/westsun0920/assignment01}{URL : https://github.com/westsun0920/assignment01}
%----------------------------------------------------------------------------------------
%	PROBLEM 1
%----------------------------------------------------------------------------------------


\section{Creating account and making a project} % Numbered section

To use the github, first create an account using your ID, email, and password.
When you log in after creating an account, you create a new project by pressing the "Start a project" button on the main screen. 
Set the project name to "assignment01".
Enter "alignment01" in the "README.md" file and save it.
Then you can see the page as shown in Figure~\ref{fig:main page}.


\begin{figure}[h]
	\centering
	\includegraphics[scale=0.5]{11}
	\caption{main page}
	\label{fig:main page}
\end{figure}



%------------------------------------------------




%----------------------------------------------------------------------------------------
%	PROBLEM 2
%----------------------------------------------------------------------------------------

\section{Saving and Converting Files}

\subsection{Saving files from github to my computer}
Install the git program first and run the console to save the file stored in the github to my computer.
Then enter the following command to set up the user name and email.

\begin{itemize}
	\item git config --global user.name "westsun0920"
	\item git config --global user.email "westsun0920@gmail.com"
\end{itemize}

Then enter the following command to save the file to my computer as shown in Figure~\ref{fig:save}.

\begin{itemize}
	\item git clone git://github.com/westsun0920/assignment01
\end{itemize}

\begin{figure}[h]
	\centering
	\includegraphics[scale=1.0]{save}
	\caption{saving command}
	\label{fig:command}
\end{figure}


\begin{figure}[h]
	\centering
	\includegraphics[scale=1.0]{saveresult}
	\caption{save result}
	\label{fig:save result}
\end{figure}

As shown in Figure~\ref{fig:save result}, you can check that the file of github is stored on my computer.


\subsection{Convert files on my computer upload to github}
First, modify the "README.md" file stored on my computer using the notepad.
Enter the command as shown in Figure~\ref{fig:convert}.

\newpage

\begin{figure}[h]
	\centering
	\includegraphics[scale=1.0]{convert}
	\caption{converting command}
	\label{fig:convert}
\end{figure}

\begin{figure}[h]
	\centering
	\includegraphics[scale=0.5]{convertresult}
	\caption{converting result}
	\label{fig:convert result}
\end{figure}

You can see that the "README.md" file has been converted through the github.

\newpage
\section{Advantage of github}
These days, most people usually use more than two computers.
People work on a variety of devices, such as a company computer, home computer, or personal laptop.
In this situation, it is very convenient that all devices are connected to a github and can share work anywhere.
In addition, it is a great advantage in that you can increase the efficiency of your work by sharing it with others.
From now on, we are planning to work on a shared project using the github in our lab.


%----------------------------------------------------------------------------------------

\end{document}
